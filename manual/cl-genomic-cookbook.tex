\documentclass[a4paper, 12pt]{article}
\usepackage{a4wide}
\usepackage{xcolor}
\usepackage{hyperref}
\renewcommand{\familydefault}{\sfdefault} % Use default sans-serif font

\usepackage{listings}

\usepackage[english]{babel}
\usepackage[latin1]{inputenc}

\lstloadlanguages{Lisp}
\lstset{numbers=left,numbersep=5pt,
  numberfirstline=true,stepnumber=5,
  backgroundcolor=\color{brown!20},
  keywordstyle=\ttfamily\color{green},
  identifierstyle=\ttfamily\color{blue}\bfseries, 
  commentstyle=\color{orange},
  stringstyle=\ttfamily\color{brown},
  keywordsprefix=:,
  showstringspaces=true,
  captionpos=b}

\begin{document}

\title{The cl-genomic Cookbook}
\author{Keith James}
\date{\today}

\maketitle

\section{Creating bio-sequences}
\label{sec:create-bioseq}

\subsection{Creating sequences from strings}
\label{sec:create-bioseq-str}

There exist three exported functions that create sequence objects from
strings; \lstinline!make-dna!, \lstinline!make-rna! and
\lstinline!make-aa!, creating DNA, RNA and amino-acid sequence objects
respectively.

Unless an \lstinline[language=Lisp]!:identifer! keyword argument is
supplied, the resulting object will be an anonymous sequence
(i.e. will have an identifier NIL and will return T if passed to the
\lstinline!anonymousp! predicate). A newly created nucleic-acid
sequence object will be double-stranded unless a
\lstinline[language=Lisp]!:num-strands! argument of 1 is supplied. The
strandedness of a sequence affects some operations. For example, it is
not possible to request an interval from the reverse-complement strand
of a such a sequence (including all amino-acid sequences and
nucleic-acid sequences explicitly marked as single-stranded).

\lstset{basicstyle=\small, language=Lisp}
\begin{lstlisting}[caption={Making DNA sequences},label=make-dna]
  (in-package :bio-sequence)
  
  ;; Create an anonymous DNA sequence
  (make-dna "gtaaaacgacggccagtg")
  ;; Use of characters illegal in DNA will cause an error
  (make-dna "guaaaacgacggccagug")
  ;; Create an identified, single-stranded DNA sequence 
  (make-dna "gtaaaacgacggccagtg" :identity 'm13-fwd :num-strands 1)
  ;; Create a DNA sequence with ambiguities
  (make-dna "gtanarygacggccagtg")
  ;; Create a 30bp virtual DNA sequence
  (make-dna nil :length 30)
\end{lstlisting}

All sequences support the full IUPAC character set. Alternatively,
sequences may be created without any residues being explicitly stated,
instead supplying a \lstinline[language=Lisp]!:length! initialization
argument. The resulting object will be a virtual sequence (i.e. will
be composed implicitly of ambiguous residues and will return T if
passed to the \lstinline!virtualp! predicate).

\end{document}
