\documentclass[a4paper, 12pt]{article}
 \usepackage{xcolor}
\usepackage{hyperref}
\renewcommand{\familydefault}{\sfdefault} % Use default sans-serif font

\usepackage{listings}

\usepackage[english]{babel}
\usepackage[latin1]{inputenc}

\lstloadlanguages{Lisp}
\lstset{frame=ltrb,framesep=5pt,
  keywordstyle=\ttfamily\color{blue},
  identifierstyle=\ttfamily\color{blue!70!red}\bfseries, 
  commentstyle=\color{red!70!yellow},
  stringstyle=\ttfamily\color{brown},
  showstringspaces=true}

\begin{document}

\title{cl-genomic : A Common Lisp library for processing genomic
  sequence data}
\author{Keith James}
\date{\today}

\maketitle

\section{Introduction}
\label{sec:intro}

cl-genomic is a portable Common Lisp library for processing genomic
data, including DNA, RNA and amino-acid sequences, and the biological
relationships between them.

\subsection{Design goals}
\label{sec:design}

\subsubsection{Uniform representation of  sequence and annotation}
\label{sec:seqrep}

The ``sequence feature'' is a common bioinformatics meme, often
represented in object-oriented programming by its own class hierarchy
that exists in parallel to the sequence class hierarchy. However, the
biological entities that these features represent are essentially
sequences. cl-genomic avoids use of the ``sequence feature'' as a
non-sequence type, substituting for them ``intervals'' that are
first-class sequences.

\subsubsection{Support for formalised biological concepts}
\label{sec:biorel}

A key goal for cl-genomic is support for formal representation of
biological concepts and their relationships. While cl-genomic may be
used without an ontology, the aim is to use the Sequence Ontology for
all non-trivial cases. In doing so, it is important to avoid
hard-coding biological knowledge into the API. For example, functions
such as \lstinline!transcript-of! or \lstinline!exon-of! are avoided
because they are only relvant and correct for a subset of biological
use cases. Instead, functions have ontology term parameters to convey
this information, when required.

\subsubsection{Limited support for bioinformatics file formats}
\label{sec:bioformats}

cl-genomic will only support bioinformatics file formats that fall
into one or more of the following categories:

\begin{itemize}
\item Are structurally and semantically well-defined
\item Are simple enough to be trivial
\item Are so ubiqitous that they cannot be avoided
\end{itemize}

There exist other toolkits (notably BioPerl) that will read and write
many formats. Supporting file formats that have been used beyond the
original designers' intent, or were poorly designed in the first
place, is a significant burden. cl-genomic does not intend to
replicate that effort.


\subsection{Alternatives}
\label{sec:alternate}

At the time of writing, no general-purpose Common Lisp library for
biological sequence processing was available. (however, see the 
\href{http://common-lisp.net/project/cl-bio}{cl-bio} project).




\end{document}
